\documentclass[a4paper]{article}
\usepackage{graphics,eurosym,latexsym}
\usepackage{listings}
%% \lstset{columns=fixed,basicstyle=\ttfamily,numbers=left,numberstyle=\tiny,stepnumber=5,breaklines=true}
\lstset{columns=fixed,basicstyle=\ttfamily,numbers=none,breaklines=true}
\usepackage{times}
\usepackage[round]{natbib}
\usepackage{hyperref}
\bibliographystyle{plainnat}
\oddsidemargin=0cm
\evensidemargin=0cm
\headheight=0cm
\headsep=0cm
\newcommand{\be}{\begin{enumerate}}
\newcommand{\ee}{\end{enumerate}}
\newcommand{\bi}{\begin{itemize}}
\newcommand{\ei}{\end{itemize}}
\newcommand{\I}{\item}
\newcommand{\ty}{\texttt}
\newcommand{\kr}{K_{\rm r}}
\newcommand{\cm}{C_{\rm M}}
\textwidth=16cm
\textheight=23cm
\begin{document}
\title{\ty{macle2go}: Annotate \ty{macle} Output}
\author{Bernhard Haubold\\\small Max-Planck-Institute for Evolutionary Biology, Pl\"on, Germany}
\maketitle
\section{Introduction}
\href{http://github.com/evolbioinf/macle}{\ty{Macle}} computes a
complexity measure, the match complexity, by sliding a window across
a given genome. A major motivation for this computation is to
investigate the relationship between sequence complexity and gene
function. For instance, we might ask
\begin{itemize}
\item Are there more genes in regions
  of high complexity than expected by chance alone?
\item Are genes in high-complexity regions enriched for certain
  functions?
\end{itemize}
\ty{Macle2go} is designed to help answer such questions.
\section{Getting Started}
\ty{Macle2go} is written in Go, so assuming a working Go
installation, you can get or update the program
\begin{verbatim}
go get -u github.com/evolbioinf/macle2go
\end{verbatim}
and install it
\begin{verbatim}
go install github.com/evolbioinf/macle2go
\end{verbatim}
Now test \ty{macle2go}
\begin{verbatim}
macle2go
\end{verbatim}
which should list its four subcommands, \ty{annotate},
\ty{enrichment}, \ty{quantile}, and \ty{version}.

\section{Tutorial}
We begin by copying the example data files to our
working directory
\begin{verbatim}
cp $GOPATH/github.com/evolbioinf/macle2go/data/*.bz2 .
\end{verbatim}
Then we uncompress them
\begin{verbatim}
bunzip2 *.bz2
\end{verbatim}
Now we are ready to work through the three major commands of
\ty{macle2go}, \ty{quantile}, \ty{annotate}, and \ty{enrichment}.

\subsection{\ty{quantile}} The match complexity, $\cm$, varies between 0 and 1. If $\cm=0$, the region is
exactly repeated elsewhere in the genome. Random sequences, on the
other hand, have an expected $\cm$ of 1. The distribution of $\cm$ for
random sequences can be modelled by a normal distribution. Using the
underlying match length distribution derived by \cite{hau09:est}, we
can thus 
compute the quantiles of $\cm$ for random sequences. But first, we list the options of \ty{quantile}
\begin{lstlisting}
macle2go quantile -h
Usage: macle2go quantile options
Example: macle2go quantile -l 2937655681 -g 0.408679 -w 20000 -p 0.05
Options:
	-g <NUM> gc-content
	-l <NUM> genome length
	-w <NUM> window length
	-p <NUM> probability
\end{lstlisting}
Then run the example command
\begin{verbatim}
macle2go quantile -l 2937655681 -g 0.408679 -w 20000 -p 0.05
\end{verbatim}
where
\begin{itemize}
\item \ty{-l} is the sequence length, 2.9 Gb
\item \ty{-g} its GC content, 0.41
\item \ty{-w} the length of the sliding window, 20 kb
\item \ty{-p} the probability mass covered up to the point to be determined, 5\%
\end{itemize}
This gives
\begin{verbatim}
macle2go quantile -l 2937655681 -g 0.408679 -w 20000 -p 0.05
# SeqLen         WinLen  P     Q                  F(Q)
2.937655681e+09  20000   0.05  0.996776307942256  52.62383289119413
\end{verbatim}
where the sequence length, window length, and probability are repeated
from the input. \ty{Q} (0.9968) is the quantile, and \ty{F(Q)} (52.62) the value of the
normal distribution at this point.

\subsection{\ty{annote}}
Again, we start by listing the options:
\begin{lstlisting}
macle2go annotate -h
Usage: macle2go annotate options [inputFiles]
Example macle2go annotate -r hsRefGene.txt -c 0.9968 -w 20000 hs_20k.mac
Options:
	-r <FILE> refGene file, e. g. http://hgdownload.soe.ucsc.edu/goldenPath/hg38/database/refGene.txt.gz
	-w <NUM> window length
	-c <NUM>  minimum complexity
	[-C <NUM> maximum complexity; default: no upper limit]
	[-I <NUM> iterations; default: 10000]
	[-s <NUM> seed for random number generator; default: system-generated]
	[-u <NUM> upstream promoter region; default: 1000]
	[-d <NUM> downstream promoter region; default: 1000]
	[-G consider whole genes; default: promoter]
\end{lstlisting}
The file \ty{hs\_20k.mac} contains complexity data calculated by
\ty{macle} for the complete human genome with 20 kb sliding
windows. List the first three lines:
\begin{verbatim}
head -n 3 hs_20k.mac 
chr1	10000	-1.0000
chr1	12000	-1.0000
chr1	14000	-1.0000
\end{verbatim}
The first column is the chromosome name, followed by the
midpoint of the sliding window and the $\cm$ value. If
$\cm=-1$, as in our example, not enough nucleotides were sequenced in that window to
compute a meaningful $\cm$-value.

Our aim is now to extract all regions with a complexity equivalent to
that of a random sequence. We have just computed the cutoff value for
this using \ty{quantile}, so \ty{annotate} should extract regions with
$\cm\ge 0.9968$. Each extracted region is annotated with the list of
genes with intersecting promoters. A promoter is defined as an
interval around the transcription start site (TSS). In \ty{annotate}
the default promoter is $\mathrm{TSS}\pm 2\mathrm{kb}$, but this can
be adjusted using the options \ty{-u} and \ty{-d}. The TSS of all
human genes are listed in the file \ty{hsRefGene.txt} available from
\begin{verbatim}
http://hgdownload.soe.ucsc.edu/goldenPath/hg38/database/refGene.txt.gz
\end{verbatim}
We can now run \ty{annotate} and
list the first 10 lines of its output:
\begin{verbatim}
macle2go annotate -r hsRefGene.txt -w 20000 -c 0.9968 hs_20k.mac | head
# W    I    O    E       O/E   P
# 700  181  343  178.67  1.92  -0.0001
# Chr  Start      End        Len    C_M     Sym
chr1   1320001    1346000    26000  0.9974  CPTP INTS11 MIR6808 TAS1R3
chr1   2298001    2324000    26000  0.9999
chr1   2496001    2536000    40000  1.0045  HES5 PANK4
chr1   3058001    3088000    30000  1.0094  LINC00982 PRDM16
chr1   3116001    3142000    26000  1.0004  MIR4251
chr1   3216001    3238000    22000  0.9973
chr1   8348001    8368000    20000  0.9968
\end{verbatim}
The first two lines summarize the result:
\begin{itemize}
\item 700 windows had $\cm\ge 0.9968$ (\ty{W}).
\item Overlapping windows were merged to yield 181 intervals (\ty{I}).
\item 339 genes were observed to intersect the 181 intervals (\ty{O}).
\item 164.58 genes were expected (\ty{E}) by repeatedly sampling 700 random
  windows. By default, the re-sampling is repeated $10^{4}$ times,
  which can be changed using the \ty{-I} option.
\item The ratio of observed to expected genes is 2.06.
\item The hypothesis that $\ty{E}\ge\ty{O}$ is tested by computing the
  frequency of observing random samples containing more than 339
  genes. The negative $P$-value is meant to be read as ``less
  than'', in our example $P<10^{-4}$.
\end{itemize}
The summary is followed by the 181 intervals identified. Each interval
is described in one line consisting of six columns:
\begin{enumerate}
\item Chromosome
\item Start position
\item End position
\item Length
\item $\cm$
\item Genes, possibly none, whose promoters intersect the region
\end{enumerate}

\subsection{\ty{enrichment}}
Next we investigate whether the sample of 339 genes just discovered is
enriched for particular functions. For this we need to link the names
of genes to functional categories. The file \ty{gene2go}, which was
obtained from
\begin{verbatim}
ftp://ftp.ncbi.nih.gov/gene/DATA/gene2go.gz
\end{verbatim}
lists this information:
\begin{verbatim}
head -n 2 gene2go
#tax_id	GeneID	GO_ID	Evidence	Qualifier	GO_term	PubMed	Category
3702	814629	GO:0005634	ISM	-	nucleus	-	Component
\end{verbatim}
``GO'' stands for ``gene ontology'', a system for describing gene
functions \citep{gen00:gen}. Notice that genes are denoted by
\ty{GeneID}, rather than symbol. So we need a second file to
connect symbols to \ty{GeneID}. This is the purpose of
\ty{Homo\_sapiens.gene\_info}, obtained from
\begin{verbatim}
ftp://ftp.ncbi.nih.gov/gene/DATA/GENE_INFO/Mammalia/Homo_sapiens.gene_info.gz
\end{verbatim}
It contains the following 16 columns:
\begin{verbatim}
head -n 1 Homo_sapiens.gene_info | tr '\t' '\n' | cat -n
     1	#tax_id
     2	GeneID
     3	Symbol
     4	LocusTag
     5	Synonyms
     6	dbXrefs
     7	chromosome
     8	map_location
     9	description
    10	type_of_gene
    11	Symbol_from_nomenclature_authority
    12	Full_name_from_nomenclature_authority
    13	Nomenclature_status
    14	Other_designations
    15	Modification_date
    16	Feature_type
\end{verbatim}
So columns 2 and 3 connect \ty{GeneID} to \ty{Symbol}:
\begin{verbatim}
cut -f 2,3 Homo_sapiens.gene_info | head -n 3
GeneID	Symbol
1	A1BG
2	A2M
\end{verbatim}
However, we need not worry about this as \ty{enrichment} automatically
connects
symbols to GO-categories via \ty{GeneID}. We begin by listing its
options:
\begin{verbatim}
macle2go enrichment -h
Usage: macle2go enrichment options [inputFiles]
Example: macle2go enrichment -i Homo_sapiens.gene_info -g gene2go -r hsRefGene.txt -c 0.9968 -w 20000 hs_20k.mac
Options:
	-r <FILE> refGene file, e. g. \
           http://hgdownload.soe.ucsc.edu/goldenPath/hg38/database/refGene.txt.gz
	-i <FILE> gene-info file, e. g. \
           ftp://ftp.ncbi.nih.gov/gene/DATA/GENE_INFO/Mammalia/Homo_sapiens.gene_info.gz
	-g <FILE> gene2go file, e. g. ftp://ftp.ncbi.nih.gov/gene/DATA/gene2go.gz
	-c <NUM>  minimum complexity
	-w <NUM>  window length
	[-C <NUM>  maximum complexity; default: no upper limit]
	[-I <NUM> iterations; default: 10000]
	[-m <NUM>  minimum number of genes per GO-category; default: 10]
	[-s <NUM> seed for random number generator; default: system-generated]
	[-u <NUM> upstream promoter region; default: 1000]
	[-d <NUM> downstream promoter region; default: 1000]
	[-G analyze whole genes; default: promoter]
\end{verbatim}
\normalsize
Next, we run the example command and save the result to a file:
\begin{lstlisting}
macle2go enrichment -i Homo_sapiens.gene_info -g gene2go -r
   hsRefGene.txt -c 0.9968 -w 20000 hs_20k.mac > enr.txt
\end{lstlisting}
We take a look at the result
\small
\begin{verbatim}
head -n 2 enr.txt
GO:0000122  49   10.07   4.87   -1.00e-04  Process\
   negative_regulation_of_transcription_by_RNA_polymerase_II
GO:0000977  28   3.21    8.72   -1.00e-04  Function\
   RNA_polymerase_II_regulatory_region_sequence-specific_DNA_binding
\end{verbatim}
\normalsize
It consists of six columns:
\begin{enumerate}
\item GO accession
\item Observed number of genes, $O$
\item Expected number of genes, $E$
\item Ratio of observed to expected genes
\item $P$-value of the null hypothesis that $E\ge O$, which is
  $<10^{-4}$ in the two examples shown
\item Description of the GO-term
\end{enumerate}
As a final step in this tutorial  we extract all categories with
$P<10^{-4}$ and sort them in reverse by the ratio of $O/E$:
\begin{verbatim}
awk '$5<0' enr.txt | sort -n -k 4 -r | head -n 5
GO:0009954  11   0.29    37.87  -1.00e-04  Process    proximal/distal_pattern_formation
GO:0009952  32   0.97    32.87  -1.00e-04  Process    anterior/posterior_pattern_specification
GO:0048704  15   0.46    32.67  -1.00e-04  Process    embryonic_skeletal_system_morphogenesis
GO:0042472  11   0.70    15.72  -1.00e-04  Process    inner_ear_morphogenesis
GO:0042475  10   0.69    14.49  -1.00e-04  Process    odontogenesis_of_dentin-containing_tooth
\end{verbatim}
\bibliography{/home/haubold/References/references}
\end{document}

